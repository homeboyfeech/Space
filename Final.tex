 %%%%%%%%%%%%%%%%%%%%%%%%%%%%%%%%%%%%%%%%%%%%%%%%%%%%%%%%%%%%%%%%%%%%%%%%%%%%%
\documentclass[12pt,a4paper]{extarticle}
\usepackage[a4paper, top=20mm, left=30mm, right=10mm, bottom=20mm]{geometry}
\usepackage[utf8]{inputenc}
\usepackage[T2A]{fontenc}
\usepackage[english,russian]{babel}
\usepackage{indentfirst}
\usepackage{misccorr}
\usepackage{graphicx}
\graphicspath{{noiseimages/}}
\usepackage{amsmath}
\usepackage{mathtext}
\usepackage{float}
\usepackage{tikz}
\usetikzlibrary{shapes,arrows}
\usepackage{listings}
\usepackage{color}
\lstdefinestyle{mystyle}{
	basicstyle=\ttfamily\footnotesize,
	breakatwhitespace=false,
	keywordstyle=\color{black},
	breaklines=true,                 
	captionpos=b,                    
	keepspaces=true,                 
	numbers=left,                    
	numbersep=5pt,                  
	showspaces=false,                
	showstringspaces=false,
	showtabs=false,                  
	tabsize=2
}
\lstset{style=mystyle}

\usepackage{titlesec}
\titleformat{\section}{\normalfont\large\bfseries}{\thesection}{1em}{}
\renewcommand{\thesection}{\arabic{section}}
\renewcommand{\theenumi}{\arabic{enumi}.}
\renewcommand{\labelenumi}{\arabic{enumi}}
%%%%%%%%%%%%%%%%%%%%%%%%%%%%%%%%%%%%%%%%%%%%%%%%%%%%%%%%%%%%%%%%%%%%%%%%%%%

\begin{document}
\begin{titlepage}
\begin{center}
\MakeUppercase{Министерство науки и высшего образования Российской Федерации} \\
\MakeUppercase{ФГБОУ ВО АЛТАЙСКИЙ ГОСУДАРСТВЕННЫЙ УНИВЕРСИТЕТ}
\vfill
\textsc{Институт цифровых технологий, электроники и физики} \\

\textsc{Кафедра вычислительной техники и электроники (ВТиЭ)}
\vfill
\textsc{Лабораторная работа №6} \\
\textbf{Разработка графической программы, моделирующую солнечную систему}
\bigskip

\end{center}
\vfill

\newlength{\ML}
\settowidth{\ML}{«\underline{\hspace{0.7cm}}» \underline{\hspace{1cm}}}
\hfill\begin{minipage}{0.5\textwidth}
Выполнили студенты 506 гр.\\
\underline{\hspace{4cm}} П.\,А.~Дмуха\\
\underline{\hspace{4cm}} В.\,С.~Маркевцев\\
\underline{\hspace{4cm}} Ф.\,С.~Калугарев\\
\end{minipage}%

\hfill\begin{minipage}{0.5\textwidth}
Проверил к.т.н, доцент каф. ВТиЭ\\
\underline{\hspace{4cm}} И.\,А.~Шмаков\\
Лабораторная работа защищена\\
«\underline{\hspace{0.7cm}}» \underline{\hspace{4cm}} \the\year~г. \\
Оценка \underline{\hspace{3.6cm}}
\end{minipage}%
\vfill

\begin{center}
Барнаул, \the\year~г.
\end{center}
\end{titlepage}
%%%%%%%%%%%%%%%%%%%%%%%%%%%%%%%%%%%%%%%%%%%%%%%%%%%%%%%%%%%%%%%%%%%%
\section*{Содержание}
\addcontentsline{toc}{section}{Foo and Bar}
1. Введение

2. Постоновка задачи

3. Чт использовали

4. Парадигма

5. Разработка

6. Выбранные параметры

7. Вывод

8. Приложение
\section*{Введение}
\addcontentsline{toc}{section}{Foo and Bar}
Участники: Дмуха Павел - лидер проекта.Задачей является задать идею проекта, написать отчет и решать вопросы по программе. Калугарев Фёдр - тестировщик. Задачей является проверка работоспособности программы и нахождением ошибок внутри нее. Маркевцев Влад - разработчик.Задачей является писать код программы.

В качестве парадигмы мы решили использовать процедурное программирование, так как оно является наиболее эфективным и простым для написания программы с использованием модуля vpython.Императивное
программирование описывает расчет с помощью последовательности команд и определяет
точную процедуру решения задачи.
\section{Глава 1}
\subsection{Постановка задачи}
Написать программу, моделирующую солнечную систему, содержащуюю информацию о планетах. Планеты должны двигаться по орбитам; cкорость движения планеты должна регулироваться; движения планет в программе должно быть взаимосвязано.
\subsection{Что использовали}
 vpython, math. Для реализации программы в 3D нам потребовался модуль vpython. VPython позволяет пользователям создавать объекты, такие как сферы и конусы, в 3D-пространстве и отображать эти объекты в окне. Это облегчает создание простых визуализаций, позволяя программистам больше сосредоточиться на вычислительном аспекте своих программ.

Модуль Math в Python.\\
Math является самым базовым математическим модулем Python. Охватывает основные математические операции, такие как сумма, экспонента, модуль и так далее. Эта библиотека не используется при работе со сложными математическими операциями, такими как умножение матриц. Расчеты, выполняемые с помощью функций библиотеки math, также выполняются намного медленнее. Тем не менее, эта библиотека подходит для выполнения основных математических операций.


\section{Глава 2}
\subsection{Парадигма}
Процедурный Стиль программирования — это программирование, основанное на использовании процедур:
Процедурное программирование — программирование на императивном языке, при котором последовательно выполняемые операторы можно собрать в подпрограммы, то есть более крупные целостные единицы кода, с помощью механизмов самого языка.

Процедурное программирование является отражением архитектуры традиционных ЭВМ, которая была предложена Фон Нейманом в 1940-х годах.

Теоретической моделью процедурного программирования служит абстрактная вычислительная система под названием машина Тьюринга.

Основные сведения
Выполнение программы сводится к последовательному выполнению операторов с целью преобразования исходного состояния памяти, то есть значений исходных данных, в заключительное, то есть в результаты.

Таким образом, с точки зрения программиста имеются программа и память, причём первая последовательно обновляет содержимое последней.

Процедурный язык программирования предоставляет возможность программисту определять каждый шаг в процессе решения задачи. Особенность таких языков программирования состоит в том, что задачи разбиваются на шаги и решаются шаг за шагом.

Используя процедурный язык, программист определяет языковые конструкции для выполнения последовательности алгоритмических шагов.
\subsection{Разработка}
Что мы сделали? Мы добавили в программу данные о планетах и о их взаимосвязи с солнцем: массу каждой планеты в килограммах, среднее расстояние от Солнца до каждой из планет, записвали формулы гравитационной силы между солнцем и планетами и уголовую скорость для каждой планеты. Создали сами планеты с их радиусом и поставили их внужном порядке.Сделали так, чтобы они врощались вокруг солнца.
\label{sec:longtermgoals}
\subsection{Выбранные параметры}
\begin{itemize}
\item 1.9885e30   масса Солнца, кг
\item 5.97e24   масса Земли, кг
\item 3.33e23  масса Меркурия,кг
\item 1.89e27  масса Юпитера, кг
\item 1.02e26  масса Нептуна, кг
\item 5.68E26  масса Сатурна, кг
\item 8917  масса Урана, кг
\item 6.42e24  масса Марса, кг
\item 4.87e24  масса Венеры, кг
\item 1.08E11 среднее растояние от Солнца до Венеры, метры 
\item  2.28e11 среднее растояние от Солнца до Марса, метры 
\item  1.496e11   среднее расстояние от Солнца до Земли, метры
\item  0.58e11  среднее расстояние от Солнца до Меркурия, метры
\item  7.8e11   среднее расстояние от Солнца до Юпитера, метры
\item  4.55e12   среднее расстояние от Солнца до Нетуна, метры
\item  2.8E12  среднее расстояние от Солнца до Урана, метры
\item  1.42E12  среднее расстояние от Солнца до Сатурана, метры
\end{itemize}
\label{sec:longtermgoals}

\section{Вывод}
В ходе лабораторной работы научились работать в команде на платформе github, научились пользоваться модулем vpython для разработки графической программы. Написали программу моделирующую солнечную систему. 


\newpage
\section*{Приложение}
\subsection*{Текст программы:}
\begin{lstlisting}[language=python]
from vpython import sphere, vector, color, rotate
import math


G = 6.667e-11 
MS = 1.9885e30  
ME = 5.97e24  
Mm = 3.33e23 
MY = 1.89e27
MN = 1.02e26 
Ms = 5.68E26
MU = 8917 
MM = 6.42e24 
MV = 4.87e24
RSV = 1.08E11 
RSM = 2.28e11 
RSE = 1.496e11  
RSm = 0.58e11 
RSY = 7.8e11 
RSN = 4.55e12 
RSU = 2.8E12 
RSs = 1.42E12 


F_SU = G*MS*MU/(RSU*RSU)

F_Ss = G*MS*Ms/(RSs*RSs)

F_SY = G*MS*MY/(RSY*RSY)

F_SN = G*MS*MN/(RSN*RSN)

F_Sm = G*MS*Mm/(RSm*RSm)

F_SE = G*MS*ME/(RSE*RSE)

F_SM = G*MS*MM/(RSM*RSM)

F_SV = G*MS*MV/(RSV*RSV)



wy = 10*math.sqrt(F_SY/(MY*RSY))

wn = 10*math.sqrt(F_SN/(MN*RSN))

wu = 10*math.sqrt(F_SU/(MU*RSU))

wS = 10*math.sqrt(F_Ss/(Ms*RSs))

wM = 10*math.sqrt(F_Sm/(Mm*RSm))

we = 10*math.sqrt(F_SE/(ME*RSE))

wm = 10*math.sqrt(F_SM/(MM*RSM))

wv = 10*math.sqrt(F_SV/(MV*RSV))



v = vector(0.5, 0, 0)
Ypiter = sphere(pos=vector(6,0,0), color=color.white,radius=.40, make_trail=True)
Uran = sphere(pos=vector(8,0,0), color=color.blue,radius=.20, make_trail=True)
saturn = sphere(pos=vector(7,0,0), color=color.red,radius=.35, make_trail=True)
Neptun = sphere(pos=vector(9,0,0), color=color.blue,radius=.19, make_trail=True)
Merkury = sphere(pos=vector(2,0,0), color=color.orange,radius=.18, make_trail=True)
Venera = sphere(pos=vector(3,0,0), color=color.white,radius=.23, make_trail=True)
Mars = sphere(pos=vector(5,0,0), color=color.red,radius=.25, make_trail=True)
Eearth = sphere(pos=vector(4, 0, 0), color=color.blue, radius=.25, make_trail=True)
Sun = sphere(pos=vector(0, 0, 0), color=color.yellow, radius=1)

dt = 10

theta_earth = we*dt
theta_ypiter = wy*dt
theta_neptun = wn*dt
theta_uran = wu*dt
theta_saturn = wS*dt
theta_merkury = wM*dt
theta_mars = wm*dt
theta_venera = wv*dt
while dt <= 86400*365:
	Ypiter.pos = rotate(Ypiter.pos, angle=theta_ypiter)
	Neptun.pos = rotate(Neptun.pos, angle=theta_neptun)
	Uran.pos = rotate(Uran.pos, angle=theta_uran)
	saturn.pos = rotate(saturn.pos, angle=theta_saturn)
	Merkury.pos = rotate(Merkury.pos, angle=theta_merkury)
	Mars.pos = rotate(Mars.pos, angle=theta_mars)
	Venera.pos = rotate(Venera.pos, angle=theta_venera) 
	Eearth.pos = rotate(Eearth.pos, angle=theta_earth,)
\end{lstlisting}
\subsection*{Ссылки на использованную литературу}
--------------------------------------------------------------------------------\\
\href{https://en.wikipedia.org/wiki/VPython}\\
\href{https://bookflow.ru/matematicheskie-biblioteki-python/}\\
\href{https://habr.com/ru}\\
\subsection*{Ссылка на гит}
\href{https://github.com/homeboyfeech/Space}
\end{document}
