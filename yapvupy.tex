 %%%%%%%%%%%%%%%%%%%%%%%%%%%%%%%%%%%%%%%%%%%%%%%%%%%%%%%%%%%%%%%%%%%%%%%%%%%%%
\documentclass[12pt,a4paper]{extarticle}
\usepackage[a4paper, top=20mm, left=30mm, right=10mm, bottom=20mm]{geometry}
\usepackage[utf8]{inputenc}
\usepackage[T2A]{fontenc}
\usepackage[english,russian]{babel}
\usepackage{indentfirst}
\usepackage{misccorr}
\usepackage{graphicx}
\usepackage{amsmath}
\usepackage{mathtext}
\usepackage{float}
\usepackage{tikz}
\usetikzlibrary{shapes,arrows}
\usepackage{listings}
\usepackage{color}
\lstdefinestyle{mystyle}{
	basicstyle=\ttfamily\footnotesize,
	breakatwhitespace=false,
	keywordstyle=\color{black},
	breaklines=true,                 
	captionpos=b,                    
	keepspaces=true,                 
	numbers=left,                    
	numbersep=5pt,                  
	showspaces=false,                
	showstringspaces=false,
	showtabs=false,                  
	tabsize=2
}
\lstset{style=mystyle}

\usepackage{titlesec}
\titleformat{\section}{\normalfont\large\bfseries}{\thesection}{1em}{}
\renewcommand{\thesection}{\arabic{section}.}
\renewcommand{\theenumi}{\arabic{enumi}}
\renewcommand{\labelenumi}{\arabic{enumi}.}
%%%%%%%%%%%%%%%%%%%%%%%%%%%%%%%%%%%%%%%%%%%%%%%%%%%%%%%%%%%%%%%%%%%%%%%%%%%

\begin{document}
\begin{titlepage}
\begin{center}
\MakeUppercase{Министерство науки и высшего образования Российской Федерации} \\
\MakeUppercase{ФГБОУ ВО АЛТАЙСКИЙ ГОСУДАРСТВЕННЫЙ УНИВЕРСИТЕТ}
\vfill
\textsc{Институт цифровых технологий, электроники и физики} \\

\textsc{Кафедра вычислительной техники и электроники (ВТиЭ)}
\vfill
\textsc{Лабораторная работа №6} \\
\textbf{Разработка графической программы, моделирующую солнечную систему}
\bigskip

\end{center}
\vfill

\newlength{\ML}
\settowidth{\ML}{«\underline{\hspace{0.7cm}}» \underline{\hspace{1cm}}}
\hfill\begin{minipage}{0.5\textwidth}
Выполнил студент 506 гр.\\
\underline{\hspace{4cm}} П.\,А.~Дмуха\\
\end{minipage}%

\hfill\begin{minipage}{0.5\textwidth}
Проверил к.т.н, доцент каф. ВТиЭ\\
\underline{\hspace{4cm}} И.\,А.~Шмаков\\
Лабораторная работа защищена\\
«\underline{\hspace{0.7cm}}» \underline{\hspace{4cm}} \the\year~г. \\
Оценка \underline{\hspace{3.6cm}}
\end{minipage}%
\vfill

\begin{center}
Барнаул, \the\year~г.
\end{center}
\end{titlepage}
%%%%%%%%%%%%%%%%%%%%%%%%%%%%%%%%%%%%%%%%%%%%%%%%%%%%%%%%%%%%%%%%%%%%

\section{Глава 1}
\subsection{Постановка задачи}

\subsection{Что использовали}
Pygame, math. pygame это набор модулей языка программирования Python, предназначенный для написания компьютерных игр и мультимедиа-приложений. Pygame базируется на мультимедийной библиотеке SDL.\\
В сердце каждой игры лежит цикл, который принято называть «игровым циклом». Он запускается снова и снова, делая все, чтобы работала игра. Каждый цикл в игре называется кадром.

В каждом кадре происходит масса вещей, но их можно разбить на три категории:

Обработка ввода (события)
Речь идет обо всем, что происходит вне игры — тех событиях, на которые она должна реагировать. Это могут быть нажатия клавиш на клавиатуре, клики мышью и так далее.

Обновление игры
Изменение всего, что должно измениться в течение одного кадра. Если персонаж в воздухе, гравитация должна потянуть его вниз. Если два объекта встречаются на большой скорости, они должны взорваться.

Рендеринг (прорисовка)
В этом шаге все выводится на экран: фоны, персонажи, меню. Все, что игрок должен видеть, появляется на экране в нужном месте.

Время

Еще один важный аспект игрового цикла — скорость его работы. Многие наверняка знакомы с термином FPS, который расшифровывается как Frames Per Second (или кадры в секунду). Он указывает на то, сколько раз цикл должен повториться за одну секунду. Это важно, чтобы игра не была слишком медленной или быстрой. Важно и то, чтобы игра не работала с разной скоростью на разных ПК. Если персонажу необходимо 10 секунд на то, чтобы пересечь экран, эти 10 секунд должны быть неизменными для всех компьютеров.


Модуль Math в Python.
Python библиотека math содержит наиболее применяемые математические функции и константы. Все вычисления происходят на множестве вещественных чисел. 


\section{Глава 2}
\subsection{Парадигма}

\subsection{Выбранные параметры}

\section{Глава 3. Вывод}



\newpage
\section*{Приложение}
\subsection*{Текст программы:}
\begin{lstlisting}[language=C++]

\end{lstlisting}
\subsection*{Ссылка на гит}
\end{document}